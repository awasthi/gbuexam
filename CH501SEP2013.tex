\documentclass[11pt]{article}
\usepackage{pslatex}
\usepackage{exam,amssymb}
\pagestyle{fancy}
\newcommand{\pd}[2]{\frac{\partial (#1)}{\partial (#2)}}
\newcommand{\Jaco}[2]{\frac{\partial (#1)}{\partial (#2)}}
\newcommand{\F}{\mathbb{F}}
\newcommand{\R}{\mathbb{R}}
\newcommand{\C}{\mathbb{C}}
\paper{Roll No.
	%\centering
		\begin{tabular}{|c|c|c|c|c|c|c|c|}
		\hline
			& & & & & & &  \\
			\hline
		\end{tabular}
}
%\rollnobox
\title{First Semester}
\subject{M.Sc. Applied Mathematics}
\coursename{Computational Methods in Chemistry}
\coursecode{~CH-501}
\maxmarks{50}
\time{2:00}
\semester{Mid Semester}
\year{September, 2013}
%\answersheettop
%\subjecttwo{Paper Id: 0934}
%\note{Attempt \textbf{all} the questions.}
\begin{document} 
\begin{questions}
\question Attempt ALL parts of the following: \marks{$5 \times 2=10$}
	\begin{enumerate}
	\item There are 365.24 days in a year and exactly 1440 minutes in a day. How many significant digits are in 365.24? How many are in 1440.?
	\item Methane is a compound consisting of a 1 : 4 ratio of carbon and hydrogen atoms. (a) If a sample of methane contains 1565 atoms, how many carbon and hydrogen atoms are present? (b) Can a sample of methane contain a total of 1566 atoms?
	\item Estimate the approximate pressure at 19:64 and 19:66 from the following data:
			\begin{table}[h!]
			\centering
				\begin{tabular}{c c c c c c c c}
				   Time & 19:61 & 19:62 &  19:63 &  19:64 & 19:65  & 19:66 & 19:67 \\
				 	 Pressure & 20 &  22 & 26 & - & 35 & - & 43 \\
				\end{tabular}
				\end{table}
				\vspace{-0.3cm}
	\item Find $\Delta^6(1-2x)(1-3x^2)(4-7x^3)$ and $\Delta^7(1-2x)(1-3x^2)(4-7x^3)$
 	\item Show that $\Delta \equiv E-1$

	\end{enumerate}
\vspace{-0.2cm}
\item Attempt ALL parts of the following: \marks{$2 \times 5= 10$}
\begin{enumerate}
\item Define the following terms and give a suitable example: Commutativity, Associativity, Inverse, Identity, Distributivity.
\item Define Group. Show that $\R$ (set of real numbers) with addition $(+)$ operation is a group.
\end{enumerate}
\vspace{-0.2cm}
\item Attempt ALL parts of the following: \marks{$2 \times 5= 10$}
\begin{enumerate}
\item We define factorial notation as  $x^{(n)} = x (x-1)(x-2) \ldots (x-\bar{n-1})$. Express $x^2 + 11x + 5$ in factorial notation.
	\item Define Absolute Error, Relative Error and Percentage Error. The diameter and height of a right circular cylinder are measured to be 5 and
8 cm. respectively. If each of these dimensions may be in error by $\pm 0.1$ cm, find the  percentage error in volume of the cylinder.
\end{enumerate}
\vspace{-0.2cm}
\item Attempt ALL parts of the following: \marks{$2 \times 5= 10$}
\begin{enumerate}
	\item During an experiment we have collected 7 readings, each at one hour distance, 2.3, 4.5, 5.6, 6.7, 7.8,8.9,9.1. Make a difference table and Write the values of $\Delta^2y_0$ and $\Delta^3y_1$.
	\item Given $u_0 = 3, u_1 = 12, u_2 = 81, u_3 = 200, u_4 = 100$ and $u_5 = 8$. find the value of $\Delta^5u_0$
\end{enumerate}
\vspace{-0.2cm}
\item Attempt ALL parts of the following: \marks{$2 \times 5= 10$}
\begin{enumerate}
	\item Obtain the first term of the series whose second and subsequent terms are 8,3,0,−1,0.
		\item Given $u_0 = 3, u_1 = 12, u_2 = 81, u_3 = 200, u_4 = 100$ and $u_5 = 8$. find the value of $\Delta^5u_0$
\end{enumerate}

\end{questions}
\end{document}
 
